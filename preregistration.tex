\documentclass[12pt]{article}
\usepackage[margin=1in]{geometry}
\usepackage{hyperref}
\usepackage{enumitem}
\usepackage{booktabs}

\title{OSF Preregistration: Inner Speech Stop-Signal Study}
\author{Thomas Morton \and Alexandra Pedego \and Victor Ferreira\\
\textit{University of California, San Diego --- Department of Psychology}}
\date{}

\begin{document}

\maketitle

\section*{Title}
% Provide a concise and descriptive title for your registration that accurately reflects the main focus of your research.

\section*{Description}
% Write a detailed description of your research project, including its purpose and any expected outcomes.

\section*{Contributors}
\begin{itemize}[nosep]
    \item Thomas Morton -- UCSD Psychology
    \item Alexandra Pedego -- UCSD Psychology
    \item Victor Ferreira -- UCSD Psychology
\end{itemize}

\section*{Tags}
% Enter specific keywords that describe the key elements and concepts of your research.

\section*{Hypotheses}
% List specific, concise, and testable hypotheses. Please state if the hypotheses are directional or non-directional.

\section*{Study Type}
\begin{itemize}[nosep]
    \item[\checkmark] \textbf{Experiment} -- A researcher randomly assigns treatments to study subjects, this includes field or lab experiments. This is also known as an intervention experiment and includes randomized controlled trials.
    \item[$\square$] Observational Study -- Data is collected from study subjects that are not randomly assigned to a treatment.
    \item[$\square$] Meta-Analysis -- A systematic review of published studies.
    \item[$\square$] Other
\end{itemize}

\section*{Blinding}
\begin{itemize}[nosep]
    \item[$\square$] No blinding is involved in this study.
    \item[$\square$] For studies that involve human subjects, they will not know the treatment group to which they have been assigned.
    \item[$\square$] Personnel who interact directly with the study subjects will not be aware of the assigned treatments (double blind).
    \item[$\square$] Personnel who analyze the data collected from the study are not aware of the treatment applied to any given group.
\end{itemize}

\subsection*{Additional Blinding}
% Is there any additional blinding in this study?

\section*{Study Design}

The proposed study is a stop-signal picture naming study. Subjects are shown a fixation cross on a screen, then the cross is removed and replaced with the image of a concrete object. Subjects are asked to name the object as soon as possible, they are informed that on some trials, a written distractor word in red will be overlaid on the image, and that while preparing to name the image they should assess whether the distractor word written is the same as the image on the screen. This design follows from Slevc and Ferreira (2006), however it differs in that the picture is named with internal speech, without movement of the lips or tongue. In this case, participants are similarly tasked to name the object in their head, and are instructed to press a button when they begin naming the picture in their head.

If the word differs from the image on the screen, they should attempt to avoid naming the image, or stop naming the image as soon as possible. On some trials, no word appears overlaid on the image, in which case they are to continue naming the image as if a word the same as the displayed image appeared. These make up three trial types:
\begin{itemize}[nosep]
    \item \textbf{Go trials}: No word appears
    \item \textbf{Go-signal trials}: A matching distractor word appears
    \item \textbf{Stop-signal trials}: A non-matching word appears
\end{itemize}

Stop-signal distractor words are manipulated within-subjects $2 \times 2 + 2$ following from Slevc and Ferreira (2006). The $2 \times 2$ factors are semantic relatedness and phonological relatedness, where distractor words, for instance of a picture of an apple can be:
\begin{itemize}[nosep]
    \item Semantically and phonologically related (e.g., \textit{apricot})
    \item Semantically related but not phonologically related (e.g., \textit{peach})
    \item Phonologically related but not semantically related (e.g., \textit{apathy})
    \item Unrelated to the image (e.g., \textit{couch})
\end{itemize}

This follows from Experiment 1 and 2 of Slevc and Ferreira. The additional binary factor is from Experiment 5 of Slevc and Ferreira, where distractors are manipulated between emotionally valent words (e.g., \textit{murder}) and length-matched emotionally neutral control words (e.g., \textit{agreed}).

The same 18 pictures from Slevc and Ferreira are used, specifically the adapted pictures from Snodgrass \& Vanderwart (1980) by Rossion \& Portois (2004). The experiment consists of:
\begin{itemize}[nosep]
    \item \textbf{Practice section}: 72 practice trials
    \begin{itemize}[nosep]
        \item 27 go trials
        \item 27 go-signal trials
        \item 18 stop-signal trials (using unrelated distractor words)
    \end{itemize}
    \item \textbf{Primary section}: 432 experimental trials
    \begin{itemize}[nosep]
        \item 162 go trials (37.5\%, each picture nine times)
        \item 162 go-signal trials (37.5\%, each picture nine times)
        \item 108 stop-signal trials (25\%, six times each picture $2 \times 2$: 72, six times each picture $+2$: 36)
    \end{itemize}
\end{itemize}

A 25\% percentage of stop-signal trials ensures that participants are not anticipating stops in enough trials that it influences regular naming behavior. This design follows exactly from Experiment 5 of Slevc and Ferreira, including a fixed Stop Onset Asynchrony (SOA) of 200ms with no staircase procedure.

\section*{Randomization}

Trials are pseudorandomized such that more than 2 stop-signal trials never occur adjacent to each other. Participants are given three breaks, equally spaced.

\section*{Existing Data}

No existing data has been collected before the submission of this preregistration.

\section*{Explanation of Existing Data}

Not applicable.

\section*{Data Collection Procedures}

Participants will consist of university students from the University of California, San Diego. Participants must have normal to corrected-to-normal vision and be native speakers of English. Participants will be provided with SONA credit for their participation, which can be applied for class credit.

\section*{Sample Size}
% Describe the sample size of your study.

\section*{Sample Size Rationale}
% This could include a power analysis or an arbitrary constraint such as time, money, or personnel.

\section*{Stopping Rule}

N/A

\section*{Manipulated Variables}
% Precisely define all variables you plan to manipulate and the levels or treatment arms of each variable.

\section*{Measured Variables}

\begin{itemize}[nosep]
    \item Go RT
    \item Go-Signal RT
    \item Stop Trial Halt \%
    \item Stop Trial non-stop RT
\end{itemize}

\section*{Indices}

Presence of Button Press and Button Press Reaction times are combined using the stop-signal race model into the derived measure of \textbf{stop-signal reaction time (SSRT)}.

\section*{Statistical Models}

We will use Linear Mixed Effects models:

\begin{verbatim}
Stop ~ Phonological Similarity + Semantic Similarity +
       (1 | Picture) + (1 | Trial) + (1 | Subject)
\end{verbatim}

\section*{Transformations}
% If you plan on transforming, centering, recoding the data, or requiring a coding scheme for categorical variables, please describe that process.

\section*{Inference Criteria}
% What criteria will you use to make inferences? Please describe the information you'll use.

\section*{Data Exclusion}

\begin{itemize}[nosep]
    \item Participants who report a lack of inner voice will be excluded from the sample.
    \item Participants will be asked to report whether they experience visual imagery; as it is unknown what effect this may have on the task, we reserve the ability to exclude such participants from the sample and will explain the reasoning and analysis behind this decision if so.
    \item Participants with go-signal naming times outside of two standard deviations from the overall mean will be excluded.
    \item Participants who perform the task incorrectly will be excluded.
\end{itemize}

\section*{Missing Data}

We will collect replacement participants for our sample.

\section*{Exploratory Analysis}
% If you plan to explore your data to look for unspecified differences or relationships, you may include those plans here.

\section*{Other}
% Literature cited, disclosures of any related work such as replications or work that uses the same data, or other helpful context.

\end{document}
